To enable Harf\+Buzz bindings for Python among other languages, make sure you have latest version of gobject-\/introspection available. On Ubuntu, you can install that this way\+:


\begin{DoxyCode}{0}
\DoxyCodeLine{sudo\ apt-\/get\ install\ libgirepository1.0-\/dev}

\end{DoxyCode}


And then run {\ttfamily meson setup} and make sure that {\ttfamily Introspection} is reported enabled in output.

Compile and install.

Make sure you have the installation lib dir in {\ttfamily LD\+\_\+\+LIBRARY\+\_\+\+PATH}, as needed for the linker to find the library.

Then make sure you also have {\ttfamily GI\+\_\+\+TYPELIB\+\_\+\+PATH} pointing to the resulting {\ttfamily \$prefix/lib/girepository-\/$\ast$} directory.

Make sure you have pygobject installed. Then check that the following import works in your Python interpreter\+:


\begin{DoxyCode}{0}
\DoxyCodeLine{\textcolor{keyword}{from}\ gi.repository\ \textcolor{keyword}{import}\ HarfBuzz}

\end{DoxyCode}


If it does, you are ready to call Harf\+Buzz from Python! Congratulations. See \href{src/sample.py}{\texttt{ {\ttfamily src/sample.\+py}}}.

The Python API will change. Let us know on the mailing list if you are using it, and send lots of feedback. 