Here are some benchmarks, run on a 2.\+3\+GHz 8-\/core i9 macbook pro. The input text is a 1106 KB Markdown file built by concatenating the Markdown sources of all the localizations of the first edition of \href{https://github.com/progit/progit/tree/master/en}{\texttt{ {\itshape Pro Git}}} by Scott Chacon.

\tabulinesep=1mm
\begin{longtabu}spread 0pt [c]{*{2}{|X[-1]}|}
\hline
\PBS\centering \cellcolor{\tableheadbgcolor}\textbf{ Implementation   }&\PBS\raggedleft \cellcolor{\tableheadbgcolor}\textbf{ Time (sec)    }\\\cline{1-2}
\endfirsthead
\hline
\endfoot
\hline
\PBS\centering \cellcolor{\tableheadbgcolor}\textbf{ Implementation   }&\PBS\raggedleft \cellcolor{\tableheadbgcolor}\textbf{ Time (sec)    }\\\cline{1-2}
\endhead
{\bfseries{commonmark.\+js}}   &\PBS\raggedleft 0.\+59    \\\cline{1-2}
{\bfseries{cmark}}   &\PBS\raggedleft 0.\+12    \\\cline{1-2}
{\bfseries{md4c}}   &\PBS\raggedleft 0.\+04   \\\cline{1-2}
\end{longtabu}


To run these benchmarks, use {\ttfamily make bench PROG=/path/to/program}.

{\ttfamily time} is used to measure execution speed. The reported time is the {\itshape difference} between the time to run the program with the benchmark input and the time to run it with no input. (This procedure ensures that implementations in dynamic languages are not penalized by startup time.) \mbox{\hyperlink{class_a}{A}} median of ten runs is taken. The process is reniced to a high priority so that the system doesn\textquotesingle{}t interrupt runs. 