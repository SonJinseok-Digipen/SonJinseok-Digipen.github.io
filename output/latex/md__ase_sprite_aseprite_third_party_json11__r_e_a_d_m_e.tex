json11 is a tiny JSON library for C++11, providing JSON parsing and serialization.

The core object provided by the library is \mbox{\hyperlink{classjson11_1_1_json}{json11\+::\+Json}}. \mbox{\hyperlink{class_a}{A}} Json object represents any JSON value\+: null, bool, number (int or double), string (std\+::string), array (std\+::vector), or object (std\+::map).

Json objects act like values. They can be assigned, copied, moved, compared for equality or order, and so on. There are also helper methods Json\+::dump, to serialize a Json to a string, and Json\+::parse (static) to parse a std\+::string as a Json object.

It\textquotesingle{}s easy to make a JSON object with C++11\textquotesingle{}s new initializer syntax\+: \begin{DoxyVerb}Json my_json = Json::object {
    { "key1", "value1" },
    { "key2", false },
    { "key3", Json::array { 1, 2, 3 } },
};
std::string json_str = my_json.dump();
\end{DoxyVerb}
 There are also implicit constructors that allow standard and user-\/defined types to be automatically converted to JSON. For example\+: \begin{DoxyVerb}class Point {
public:
    int x;
    int y;
    Point (int x, int y) : x(x), y(y) {}
    Json to_json() const { return Json::array { x, y }; }
};

std::vector<Point> points = { { 1, 2 }, { 10, 20 }, { 100, 200 } };
std::string points_json = Json(points).dump();
\end{DoxyVerb}
 JSON values can have their values queried and inspected\+: \begin{DoxyVerb}Json json = Json::array { Json::object { { "k", "v" } } };
std::string str = json[0]["k"].string_value();
\end{DoxyVerb}
 For more documentation see \mbox{\hyperlink{json11_8hpp_source}{json11.\+hpp}}. 